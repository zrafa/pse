\documentclass[output=paper, 
colorlinks,
citecolor=brown,
newtxmath
]{langscibook} 
\bibliography{localbibliography}

\author{}
\title{Voltage, Corriente y Resistencia - Ley de Ohm}  
\abstract{

Al comenzar a explorar el mundo de la electricidad y la electrónica, es vital comenzar por entender los conceptos básicos de voltaje, corriente y resistencia. Estos son los tres bloques básicos necesarios para manipular y utilizar la electricidad. Al principio, estos conceptos pueden ser difíciles de entender porque no podemos verlos. No podemos ver a simple vista la energía fluyendo a través de un cable o el voltaje de una batería colocada sobre una mesa. Incluso los rayos en el cielo, aunque visibles, no son realmente el intercambio de energía que ocurre entre las nubes y la tierra, sino una reacción en el aire a la energía que pasa a través de él. Para detectar esta transferencia de energía, debemos usar herramientas de medición como multímetros, analizadores de espectro y osciloscopios para visualizar qué está sucediendo con la carga en un sistema. Sin embargo, no temas, este tutorial te dará la comprensión básica del voltaje, la corriente y la resistencia y cómo los tres se relacionan entre sí.

\keywords{voltage, corriente, resistencia, amperio, ohm}
}


% add all extra packages you need to load to this file  
\usepackage{tabularx} 

%%%%%%%%%%%%%%%%%%%%%%%%%%%%%%%%%%%%%%%%%%%%%%%%%%%%
%%%                                              %%%
%%%           Examples                           %%%
%%%                                              %%%
%%%%%%%%%%%%%%%%%%%%%%%%%%%%%%%%%%%%%%%%%%%%%%%%%%%% 
%% to add additional information to the right of examples, uncomment the following line
% \usepackage{jambox}
%% if you want the source line of examples to be in italics, uncomment the following line
% \renewcommand{\exfont}{\itshape}
% \usepackage{./langsci/styles/jambox}
\usepackage{./langsci/styles/langsci-lgr}
\usepackage{./langsci/styles/langsci-osl}
\usepackage{langsci-optional}
% \usepackage{langsci-gb4e}
\usepackage{langsci-cgloss}

\usepackage[linguistics]{forest}

\makeatletter
\let\thetitle\@title
\let\theauthor\@author 
\makeatother

\newcommand{\togglepaper}[1][0]{ 
%   \bibliography{../localbibliography}
  \papernote{\scriptsize\normalfont
Esta es una obra derivada oficialmente autorizada por O'Reilly(*) para la materia "Programación de Sistemas Embebidos" de la Facultad de Informática, Universidad Nacional del Comahue. \\ Detalles de la autorización oficial y sus modificaciones al final del capítulo. \\ Libro original: Programming Embedded Systems with C and GNU Development Tools,
Second Edition, by Michael Barr and Anthony Massa. Copyright 2007 O'Reilly Media, Inc., 978-0-596-00983-0
%    \theauthor.
%    \thetitle. 
%    To appear in: 
%    Change Volume Editor \& in localcommands.tex 
%    Change volume title in localcommands.tex
 %   Berlin: Language Science Press. [preliminary page numbering]
  }
 % \pagenumbering{roman}
  \setcounter{chapter}{#1}
  \addtocounter{chapter}{-1}
}


\usepackage[spanish]{babel}
\usepackage[bottom]{footmisc}
\usepackage{tabularx}

\definecolor{aliceblue}{rgb}{0.90, 0.93, 0.96}


\begin{document}

\selectlanguage{spanish}

\chapterfont{\Large\color{LightBlue}} 
% \chapter*{Programación de Sistemas Embebidos 2020\\ Makefiles}
% {\def\addcontentsline#1#2#3{}\maketitle}
% \chapter*{capitulo 3}
% {\def\addcontentsline#1#2#3{}\maketitle}
% \chapter*{capitulo 4}
% {\def\addcontentsline#1#2#3{}\maketitle}

\setcounter{chapter}{-1}
\chapter*{Voltage, Corriente y Resistencia - Ley de Ohm} 

\begingroup
\let\clearpage\relax
\cleardoublepage
\hypersetup{linkcolor=blue}
\tableofcontents
\let\clearpage\relax
\cleardoublepage
\endgroup



{\def\addcontentsline#1#2#3{}\maketitle}


\setcounter{page}{1}


% \end{minipage}

% \togglepaper[0]

\def\maketitle{

% Titulo 
 \makeatletter
 {\color{bl} \centering \huge \sc \textbf{
\large \vspace*{-8pt} \color{black} Voltage, Corriente y Resistencia - Ley de Ohm
 \vspace*{8pt} }\par}
 \makeatother


% Autor
 \makeatletter
 {\centering \small 
 	\vspace{20pt} }
 \makeatother

}



















\subsection{Cómo se relacionan el Voltaje, la Corriente y la Resistencia}

La primera, y quizás la más importante, relación entre corriente, voltaje y resistencia se llama \textit{\textbf{Ley de Ohm}}, descubierta por Georg Simon Ohm y publicada en su artículo de 1827, Investigación Matemática del Circuito Galvánico.

La corriente eléctrica es el flujo de electrones (carga eléctrica) que circula por un conductor en un determinado momento. Este flujo de carga eléctrica se puede aprovechar para hacer algún trabajo. Encender un foco, el tele, el teléfono, etc., todos aprovechan el movimiento de los electrones para realizar su trabajo. Todos operan utilizando la misma fuente de energía básica: el movimiento de electrones.

Un \textit{\textbf{circuito eléctrico}} se forma cuando se crea un camino conductivo para permitir que la carga eléctrica se mueva continuamente. Este movimiento continuo de carga eléctrica a través de los conductores de un circuito se llama \textit{\textbf{corriente}}, y a menudo se refiere en términos de flujo, al igual que el flujo del agua a través de una manguera.

La fuerza que motiva a los portadores de carga a fluir en un circuito se llama \textit{\textbf{voltaje}}. El voltaje es una medida específica de la energía potencial que siempre es relativa entre dos puntos. Cuando hablamos de cierta cantidad de voltaje presente en un circuito, nos referimos a la medición de cuánta energía potencial existe para mover portadores de carga de un punto particular en ese circuito a otro punto particular. Sin referencia a dos puntos específicos, el término voltaje no tiene significado.

La corriente tiende a moverse a través de los conductores con cierto grado de fricción, u oposición al movimiento. Esta oposición al movimiento se llama \textit{\textbf{resistencia}}. La cantidad de corriente en un circuito depende de la cantidad de voltaje y la cantidad de resistencia en el circuito para oponerse al flujo de corriente.

Al igual que el voltaje, la resistencia es una cantidad relativa entre dos puntos. Por esta razón, las cantidades de voltaje y resistencia a menudo se declaran como entre o a través de dos puntos en un circuito.


\subsection{Coulomb y Carga Eléctrica}
Una unidad fundamental de medida eléctrica que se enseña a menudo al principio de los cursos de electrónica pero se usa infrecuentemente después, es la unidad del coulomb, que es una medida de carga eléctrica proporcional al número de electrones en un estado desequilibrado. Un coulomb de carga es igual a 6,250,000,000,000,000,000 electrones.

El símbolo para la cantidad de carga eléctrica es la letra mayúscula Q, con la unidad de coulombs abreviada por la letra mayúscula C. Resulta que la unidad de flujo de corriente, el amperio, es igual a 1 coulomb de carga que pasa por un punto dado en un circuito en 1 segundo. Expresado en estos términos, la corriente es la tasa de movimiento de carga eléctrica a través de un conductor.

Como se mencionó anteriormente, el voltaje es la medida de la energía potencial por unidad de carga disponible para motivar el flujo de corriente de un punto a otro.  Definido en estos términos científicos, 1 voltio es igual a 1 julio de energía potencial eléctrica por (dividido por) 1 coulomb de carga. Por lo tanto, una batería de 9 voltios libera 9 julios de energía por cada coulomb de carga que se mueve a través de un circuito.

¿Y qué es un julio?. La unidad métrica general para la energía de cualquier tipo es el julio, que es igual a la cantidad de trabajo realizado por una fuerza de 1 newton ejercida a través de un movimiento de 1 metro (en la misma dirección). En unidades imperiales, esto es ligeramente menos de 3/4 de libra de fuerza ejercida sobre una distancia de 1 pie. En términos comunes, se necesita alrededor de 1 julio de energía para levantar un peso de 3/4 de libra 1 pie del suelo, o para arrastrar algo a una distancia de 1 pie usando una fuerza de tracción paralela de 3/4 de libra. 


\subsection{Unidades de Medición: Voltio, Amperio y Ohmio}
Para poder hacer declaraciones significativas sobre estas cantidades en circuitos, necesitamos ser capaces de describir sus cantidades de la misma manera que podríamos cuantificar la masa, la temperatura, el volumen, la longitud o cualquier otro tipo de cantidad física. Para la masa, podríamos usar las unidades de kilogramo o gramo.

Aquí están las unidades estándar de medición para la corriente eléctrica, el voltaje y la resistencia:

\begin{figure}
\includegraphics[scale=0.5]{images/units-measurement-electrical-current.png}
\caption{Las unidades de medición para la corriente eléctrica}
\label{fig:unidades}
\end{figure}

El símbolo dado para cada cantidad es la letra alfabética estándar utilizada para representar esa cantidad en una ecuación algebraica. Letras estandarizadas como estas son comunes en las disciplinas de física e ingeniería y son reconocidas internacionalmente.

La abreviatura de la unidad para cada cantidad representa el símbolo alfabético utilizado como notación abreviada para su unidad de medición particular. Y, sí, ese símbolo parecido a un herradura es la letra griega mayúscula Ω.

Todos estos símbolos se expresan con letras mayúsculas, excepto en los casos en que una cantidad (especialmente voltaje o corriente) se describe en términos de un breve período de tiempo (llamado valor instantáneo). Por ejemplo, el voltaje de una batería, que es estable durante un largo período, se representará con la letra mayúscula E, mientras que el pico de voltaje de un rayo en el momento exacto en que golpea una línea de energía probablemente se representaría con una letra minúscula e (o v minúscula) para designar ese valor en un solo momento en el tiempo. La mayoría de las mediciones de corriente continua (CC), siendo estables en el tiempo, se representarán con letras mayúsculas.


\subsection{La Ecuación de la Ley de Ohm}
El principal descubrimiento de Ohm fue que la cantidad de corriente eléctrica a través de un conductor metálico en un circuito es directamente proporcional al voltaje aplicado a través de él, para cualquier temperatura dada. Ohm expresó su descubrimiento en forma de una ecuación simple, describiendo cómo se interrelacionan el voltaje, la corriente y la resistencia.

\[V = I \times R\]

Donde:

\begin{itemize}
  \setlength\itemsep{-0.5em}
\item V = Voltaje en voltios
\item I = Corriente en amperios 
\item R = Resistencia en ohmios
\end{itemize}

Esto se llama ley de Ohm. 

1 amperio (corriente) = 1 coulomb por segundo. Es decir, si la intensidad de la corriente eléctira es de un amperio, entonces significa que en ese punto del material conductor pasan 6,250,000,000,000,000,000 de electrones por segundo. 

1 Ohmio es la resistencia entre dos puntos en un conductor donde la aplicación de 1 voltio empujará 1 amperio, o 6.241 × 10\^18 electrones. 


Digamos, por ejemplo, que tenemos un circuito con el potencial de 1 voltio, una corriente de 1 amperio y una resistencia de 1 ohmio. Usando la ley de Ohm, podemos decir:

\[1V = 1A \times 1Ω\]


Al describir el voltaje, la corriente y la resistencia, una analogía común es un tanque de agua. En esta analogía, la carga se representa por la cantidad de agua, el voltaje se representa por la presión del agua y la corriente se representa por el flujo de agua. Así que para esta analogía, recuerda:

\begin{figure}
\includegraphics[scale=1.1]{images/ohm.png}
\caption{Analogía para comprender la Ley de Ohm}
\label{fig:unidades}
\end{figure}

\begin{itemize}
  \setlength\itemsep{-0.5em}
\item Agua = Carga
\item Presión = Voltaje
\item Flujo = Corriente
\end{itemize}

En la parte inferior de este tanque hay una manguera. El voltaje es como la presión creada por el agua. La presión en el extremo de la manguera puede representar el voltaje. El agua en el tanque representa la carga. Cuanta más agua haya en el tanque, mayor será la carga, mayor será la presión que se mida en el extremo de la manguera.

Podemos pensar en este tanque como una batería, un lugar donde almacenamos una cierta cantidad de energía y luego la liberamos. Si vaciamos nuestro tanque una cierta cantidad, la presión creada en el extremo de la manguera disminuye. Podemos pensar en esto como una disminución del voltaje, como cuando una linterna se vuelve más tenue a medida que las pilas se agotan. También hay una disminución en la cantidad de agua que fluirá a través de la manguera. Menos presión significa menos agua fluyendo, lo que nos lleva a la corriente.
Digamos que esto representa nuestro tanque con una manguera ancha. La cantidad de agua en el tanque se define como 1 voltio y la estrechez (resistencia al flujo) de la manguera se define como 1 ohmio. Usando la Ley de Ohm, esto nos da un flujo (corriente) de 1 amperio (Figura 3 izquierda).

Usando esta analogía, ahora veamos el tanque con la manguera estrecha. Debido a que la manguera es más estrecha, su resistencia al flujo es mayor. Definamos esta resistencia como 2 ohmios. La cantidad de agua en el tanque es la misma que en el otro tanque, por lo que, usando la Ley de Ohm, nuestra ecuación para el tanque con la manguera estrecha es (Figura 3 derecha):

\begin{figure}[H]
\includegraphics[scale=1]{images/ohm2.png}
\caption{Manguera estrecha = mayor resistencia}
\label{fig:unidades}
\end{figure}

\[1V = I \times 2Ω\]


Pero, ¿cuál es la corriente? Debido a que la resistencia es mayor y el voltaje es el mismo, esto nos da un valor de corriente de 0.5 amperios:

\[0.5A = 1V / 2Ω\]

Entonces, la corriente es menor en el tanque con mayor resistencia. Ahora podemos ver que si conocemos dos de los valores para la ley de Ohm, podemos resolver el tercero. Demostremos esto con un experimento.

\subsection{Un Experimento de la Ley de Ohm}

Para este experimento, queremos usar una batería de 9 voltios para alimentar un LED. Los LEDs son frágiles y solo pueden tener cierta cantidad de corriente fluyendo a través de ellos antes de quemarse. En la documentación de un LED, siempre habrá una clasificación de corriente. Esta es la cantidad máxima de corriente que puede pasar a través del LED en particular antes de quemarse.

\subsubsection{Materiales Necesarios}

\begin{itemize}
  \setlength\itemsep{-0.5em}
\item Un multímetro
\item Una batería de 9 voltios
\item Una resistencia de 560 ohmios (o el valor más cercano)
\item Un LED
\end{itemize}

NOTA: Los LEDs son lo que se conoce como dispositivos no ohmicos. Esto significa que la ecuación para la corriente que fluye a través del LED en sí no es tan simple como V=IR. El LED introduce algo llamado caída de voltaje en el circuito, lo que cambia la cantidad de corriente que circula a través de él. Sin embargo, en este experimento simplemente estamos tratando de proteger al LED del exceso de corriente, por lo que vamos a ignorar las características de corriente del LED y elegir el valor de la resistencia usando la Ley de Ohm para asegurarnos de que la corriente a través del LED esté seguramente por debajo de 20 mA.

Para este ejemplo, tenemos una batería de 9 voltios y un LED rojo con una clasificación de corriente de 20 miliamperios, o 0.020 amperios. Para estar seguros, preferiríamos no hacer funcionar el LED a su corriente máxima, sino más bien a su corriente sugerida, que se lista en su hoja de datos como 18 mA, o 0.018 amperios. Si simplemente conectamos el LED directamente a la batería, los valores para la ley de Ohm se verían así:

\[0.018A = 9V / 0Ω\]

¡Dividir por cero nos da una corriente infinita! Bueno, no infinita en la práctica, pero tanta corriente como la batería pueda suministrar. Dado que NO queremos que tanta corriente fluya a través de nuestro LED, vamos a necesitar una resistencia.  Podemos usar la Ley de Ohm para determinar el valor de la resistencia que nos dará el valor de corriente deseado:

\[R = V / I\]
\[R = 9V / 0.018A\]
\[R = 500Ω\]

Entonces, necesitamos un valor de resistencia de alrededor de 500 ohmios para mantener la corriente a través del LED por debajo de la clasificación de corriente máxima. 500 ohmios no es un valor común para resistencias de estantes, por lo que este dispositivo usa una resistencia de 560 ohmios en su lugar. Así es como se ve nuestro dispositivo completamente ensamblado.

 
\begin{figure}[H]
\includegraphics[scale=1]{images/circuito-led.png}
\caption{Circuito Eléctrico básico, con una resistencia y un LED}
\label{fig:unidades}
\end{figure}



¡Éxito!. Hemos elegido un valor de resistencia lo suficientemente alto como para mantener la corriente a través del LED por debajo de su clasificación máxima, pero lo suficientemente bajo como para que la corriente sea suficiente para mantener el LED bien brillante.

Este ejemplo de LED/resistencia limitadora de corriente es una ocurrencia común en la electrónica de hobby. A menudo necesitarás usar la Ley de Ohm para cambiar la cantidad de corriente que fluye a través del circuito. 

\subsection{¿Limitar la Corriente Antes o Después del LED?}

Para complicar un poco las cosas, puedes colocar la resistencia limitadora de corriente en cualquier lado del LED, ¡y funcionará igual!

Es debido a la Ley de Voltaje de Kirchoff que la resistencia limitadora de corriente puede colocarse en cualquier lado del LED y aún así tener el mismo efecto. No analizaremos aquí tal Ley.

\subsection{Análisis de Circuitos Simples con la Ley de Ohm}
Veamos cómo podrían funcionar estas ecuaciones para ayudarnos a analizar circuitos simples:

\begin{figure}[H]
\includegraphics[scale=1]{images/test.png}
\caption{Circuitos sencillos}
\label{fig:unidades}
\end{figure}


En el circuito anterior (a), solo hay una fuente de voltaje (la batería, a la izquierda) y solo una fuente de resistencia a la corriente (la lámpara, a la derecha). Esto hace que sea muy fácil aplicar la Ley de Ohm. Si conocemos los valores de dos de las tres cantidades (voltaje, corriente y resistencia) en este circuito, podemos usar la Ley de Ohm para determinar la tercera.

\begin{itemize}
  \setlength\itemsep{-0.5em}
\item En la Figura 5 (b) calcularemos la cantidad de corriente (I) en un circuito, dados los valores de voltaje (E) y resistencia (R): ¿Cuál es la cantidad de corriente (I) en este circuito?
\item En la Figura 5 (c) calcularemos la cantidad de resistencia (R) en un circuito, dados los valores de voltaje (E) y corriente (I): ¿Cuál es la cantidad de resistencia (R) ofrecida por la lámpara?
\item En la Figura 5 (d) el último ejemplo, calcularemos la cantidad de voltaje suministrada por una batería, dados los valores de corriente (I) y resistencia (R): ¿Cuál es la cantidad de voltaje proporcionada por la batería?
\end{itemize}


\subsection{Técnica del Triángulo de la Ley de Ohm}

La Ley de Ohm es una herramienta muy simple y útil para analizar circuitos eléctricos. Se utiliza tan a menudo en el estudio de la electricidad y la electrónica que debe ser memorizada por el estudiante serio. Para aquellos que aún no están cómodos con el álgebra, hay un truco para recordar cómo resolver una cantidad dada, dadas las otras dos.

Primero, coloca las letras V, I y R en un triángulo como este:

\begin{figure}[H]
\includegraphics[scale=1]{images/triangulo.png}
\caption{Técnica del triángulo de la Ley de Ohm}
\label{fig:unidades}
\end{figure}



Si conoces V e I, y deseas determinar R, simplemente elimina R de la imagen y mira lo que queda:

Si conoces V y R, y deseas determinar I, elimina I y mira lo que queda:

Por último, si conoces I y R, y deseas determinar V, elimina V y mira lo que queda:

Eventualmente, tendrás que estar familiarizado con el álgebra para estudiar seriamente la electricidad y la electrónica, pero este consejo puede hacer que tus primeros cálculos sean un poco más fáciles de recordar. Si te sientes cómodo con el álgebra, ¡todo lo que necesitas hacer es memorizar V=IR y derivar las otras dos fórmulas a partir de eso cuando las necesites!

\subsection{REVISIÓN:}

\begin{itemize}
  \setlength\itemsep{-0.5em}
\item El voltaje se mide en voltios, simbolizado por las letras E o V.
\item La corriente se mide en amperios, simbolizada por la letra I.
\item La resistencia se mide en ohmios, simbolizada por la letra R.
\item Ley de Ohm: E = IR; I = E/R; R = E/I
\end{itemize}


\subsection{Bibliografia:}

\begin{enumerate}
  \setlength\itemsep{-0.5em}
\item Ohm’s Law - How Voltage, Current, and Resistance Relate \url{https://www.allaboutcircuits.com/textbook/direct-current/chpt-2/voltage-current-resistance-relate/}
\item Voltage, Current, Resistance, and Ohm's Law \url{https://learn.sparkfun.com/tutorials/voltage-current-resistance-and-ohms-law/all}
\end{enumerate}

\end{document}
